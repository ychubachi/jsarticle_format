\documentclass[a4j, 12Q, twocolumn, twoside]{jsarticle}
\usepackage{calc} % 数値演算を可能にするパッケージを利用する
\usepackage[
  top=30truemm, inner=24truemm, % 天とノドを設定する
  text={(27zw + 2zw + 27zw), 48\baselineskip}, % 第1引数で横幅,第2引数で縦幅を設定する
%  bottom=20truemm, outer=18truemm, % 直接指定する場合(コメントアウトを外し,こちらの場合と比較されたい)
  dvipdfm]{geometry}

\title{\LaTeX の\texttt{jsarticle}スタイルを用いた印刷仕様の設定例}
\author{中鉢欣秀\thanks{産業技術大学院大学}}
\begin{abstract}
  \LaTeX のスタイルである\texttt{jsarticle}クラスを用いA4縦の用紙に文字の大きさ12Q(ほぼ8.5pt),
  \texttt{twocolumn},\texttt{twoside}でレイアウトを設定する例.
  紙面の余白が天30mm,ノド24mm,地20mm以内,小口18mm以内に収まる版面の大きさを設定する.
  地と小口の長さを直接指定するのではなく,文字数と行数にもとづいて間接的に設定することにより,自然な字間が得られる.


\end{abstract}

\pagestyle{empty}
\begin{document}
\maketitle
\section{印刷仕様}

  印刷仕様は仕上がりがA4縦の見開きとし,余白は
  天(\texttt{top})30mm,ノド(\texttt{inner})24mmとし,
  地(\texttt{bottom})が20mm以内,小口(\texttt{outor})が18mm以内に収まるように版面を設定する.
  文字の大きさは12Qとし.段数は2段組で段間は2文字とする.

\section{jsarticleの12Q設定}
  \LaTeX における$1 pt = \frac{1}{72.27} inch$である.また,$1 inch = 25.4 mm$である.
  \texttt{jsarticle}クラスで12Qオプションを設定した場合の拡大率は$\frac{\the\mag}{1000}$である.

  まず,文字の大きさ12Q(H)であり,1Q(H) = 0.25mmであるから,
  \[ 12 \times 0.25 mm = 3 mm \]
  となる.

  行送りは\the\baselineskip であるから,
  \[ 16 \times \frac{1}{72.27} inch \times 25.4 mm/inch \times \frac{923}{1000} \approx 5.190 mm \]
  となる.よって,行間は
  \[ 5.190 mm - 3 mm = 2.190 mm \]
  と求められる.これは約8.761Hである.
  
  A4の横は$210mm$であるから,左右の余白を引いた版面の横幅は
  $210mm - 24mm - 18mm = 168mm$以内になればよい.文字の横幅は3mmであるから,56文字分に相当する.
  2段組で段間は2文字であるから各段横27字にすると\[ 27 + 2 + 27 = 56 \]となり,版面の横幅と等しくなる.
  
  次に,上下の余白を引いた版面の縦幅は$297mm - 30mm - 20mm = 247mm$以内になればよい.
  最後の行の行間$2.190mm$は不要であるので許容される縦幅に予め加え,行送り$5.190mm$で割ると
  \[ (247mm + 2.190mm) \div 5.190mm \approx 48.013 \]
  となり,48行とすれば縦幅に収まることになる.改めて,
  \[ 5.190mm \times 48 - 2.190mm = 246.93mm \]
  であるから,$297mm -30mm - 246.93 = 20.07mm$が地の部分の余白となる
  \footnote{Adobe InDesignで12Q,行間8.761Hで設定したところ地は20.058mmとなった.
  小数点のまるめに起因する誤差と考える.}
  \footnote{footnoteをつけると,地が少なくなる問題が発生しており,今後の課題である.}. 
\section{寿限無寿限無五劫の摺り切れ海砂利水魚の水行末雲来末風来末.}
寿限無寿限無五劫の摺り切れ海砂利水魚の水行末雲来末風来末.食う寝る所に住む所藪柑子ブラコウジ.パイポパイポパイポのシューリンガングーリンダイのポンポコピーのポンポコナーの長久命の長助.
寿限無寿限無五劫の摺り切れ海砂利水魚の水行末雲来末風来末.食う寝る所に住む所藪柑子ブラコウジ.パイポパイポパイポのシューリンガングーリンダイのポンポコピーのポンポコナーの長久命の長助.

寿限無寿限無五劫の摺り切れ海砂利水魚の水行末雲来末風来末.食う寝る所に住む所藪柑子ブラコウジ.パイポパイポパイポのシューリンガングーリンダイのポンポコピーのポンポコナーの長久命の長助.
\section{五劫の摺り切れ}
\subsection{海砂利水魚の水行末雲来末風来末}
寿限無寿限無五劫の摺り切れ海砂利水魚の水行末雲来末風来末.食う寝る所に住む所藪柑子ブラコウジ.
パイポパイポパイポのシューリンガングーリンダイのポンポコピーのポンポコナーの長久命の長助.
\subsection{食う寝る所に住む所藪柑子ブラコウジ}
寿限無寿限無五劫の摺り切れ海砂利水魚の水行末雲来末風来末.食う寝る所に住む所藪柑子ブラコウジ.
\begin{itemize}
  \item 寿限無寿限無 
  \item 五劫の擦り切れ
  \item 海砂利水魚の水行末雲来末風来末
  \item 寿限無寿限無五劫の摺り切れ海砂利水魚の水行末雲来末風来末
\end{itemize}
%
パイポパイポパイポのシューリンガングーリンダイのポンポコピーのポンポコナーの長久命の長助.
\begin{enumerate}
  \item パイポパイポパイポのシューリンガン
  \item グーリンダイのポンポコピーのポンポコナーの長久命の長助.
  \item パイポパイポパイポのシューリンガングーリンダイのポンポコピーのポンポコナーの長久命の長助.
\end{enumerate}
%
寿限無寿限無五劫の摺り切れ海砂利水魚の水行末雲来末風来末.食う寝る所に住む所藪柑子ブラコウジ.パイポパイポパイポのシューリンガングーリンダイのポンポコピーのポンポコナーの長久命の長助.

\section{海砂利水魚の水行末雲来末風来末}
寿限無寿限無五劫の摺り切れ海砂利水魚の水行末雲来末風来末.食う寝る所に住む所藪柑子ブラコウジ.パイポパイポパイポのシューリンガングーリンダイのポンポコピーのポンポコナーの長久命の長助.

寿限無寿限無五劫の摺り切れ海砂利水魚の水行末雲来末風来末.食う寝る所に住む所藪柑子ブラコウジ.パイポパイポパイポのシューリンガングーリンダイのポンポコピーのポンポコナーの長久命の長助.

寿限無寿限無五劫の摺り切れ海砂利水魚の水行末雲来末風来末.食う寝る所に住む所藪柑子ブラコウジ.パイポパイポパイポのシューリンガングーリンダイのポンポコピーのポンポコナーの長久命の長助.

寿限無寿限無五劫の摺り切れ海砂利水魚の水行末雲来末風来末.食う寝る所に住む所藪柑子ブラコウジ.パイポパイポパイポのシューリンガングーリンダイのポンポコピーのポンポコナーの長久命の長助.

寿限無寿限無五劫の摺り切れ海砂利水魚の水行末雲来末風来末.食う寝る所に住む所藪柑子ブラコウジ.パイポパイポパイポのシューリンガングーリンダイのポンポコピーのポンポコナーの長久命の長助.

\section{文字数・行数の確認}
\noindent
1234567890
1234567890
1234567890
\newpage
\noindent 1\par \noindent 2\par \noindent 3\par \noindent 4\par \noindent 5\par
\noindent 6\par \noindent 7\par \noindent 8\par \noindent 9\par \noindent 0\par
\noindent 1\par \noindent 2\par \noindent 3\par \noindent 4\par \noindent 5\par
\noindent 6\par \noindent 7\par \noindent 8\par \noindent 9\par \noindent 0\par
\noindent 1\par \noindent 2\par \noindent 3\par \noindent 4\par \noindent 5\par
\noindent 6\par \noindent 7\par \noindent 8\par \noindent 9\par \noindent 0\par
\noindent 1\par \noindent 2\par \noindent 3\par \noindent 4\par \noindent 5\par
\noindent 6\par \noindent 7\par \noindent 8\par \noindent 9\par \noindent 0\par
\noindent 1\par \noindent 2\par \noindent 3\par \noindent 4\par \noindent 5\par
\noindent 6\par \noindent 7\par \noindent 8\par \noindent 9\par \noindent 0\par
\noindent 1\par %\noindent 2\par \noindent 3\par \noindent 4\par \noindent 5\par

\end{document}
