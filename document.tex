\documentclass[a4j, 12Q, twocolumn, twoside]{jsarticle}
% 余白の設定
\usepackage{calc} % 数値演算を可能にするパッケージを利用する
\usepackage[
  top=30truemm, inner=24truemm, % 天とノドを設定する
  %bottom=19.80truemm, outer=18truemm, % 48行の場合
  bottom=24.99truemm, outer=18truemm, % 47行の場合
  dvipdfm]{geometry}

% 行間の設定
% \renewcommand{\baselinestretch}{1.0115} % 四分二分あきにする場合
% \renewcommand{\baselinestretch}{0.8670} % 二分あきにする場合(ルビは使わないものとする)

\title{\LaTeX の\texttt{jsarticle}スタイルを用いた印刷仕様の設定例}
\author{中鉢欣秀\thanks{産業技術大学院大学}}
\begin{abstract}
  \LaTeX のスタイルである\texttt{jsarticle}クラスを用いA4縦の用紙に文字の大きさ12Q(ほぼ8.5pt),
  \texttt{twocolumn},\texttt{twoside}でレイアウトを設定する例.
  紙面の余白を天30mm,ノド24mm,小口18mmとし,地を20mm程度確保する版面の大きさを計算した.
  その結果,行数48行ならば地を19.80mm,47行とするならば地を24.99mmとするのが最も適切な設定であることがわかった.
\end{abstract}

\pagestyle{empty}
\begin{document}
\maketitle
\section{はじめに}
  論文の原稿においてはだいたい12Qないしは8.5pt程度の文字の大きさを使用することが多い.
  \LaTeX のjsarticle クラスファイルはオプションでどちらも対応できる.
  しかし,実際に9ptオプションを設定すると,フォントサイズは約8.4ptとなる.
  
  MS Wordでも同じレイアウトを得ようとすると,フォントのサイズに8.4ptは指定できず,
  8.5ptを用いざるを得ない.そうすると,\LaTeX と同じ行数を設定した時に,行間が若干つまってしまう.
  
  jsarticleの12Qオプションは,計算するとフォントのサイズがほぼ8.5ptとなる.
  そこで,12Qオプションを利用してWordと同じレイアウトを作れないかと考えた.
  
  ここで,jsarticleでは文字の大きさは12Qになっても行間は16ptの$\frac{\the\mag}{1000}$倍となる.
  そこで,この行間において最適な行数を求めることにした.
\section{印刷仕様}

  印刷仕様は仕上がりがA4縦の見開きとし,文字
  の大きさは12Qとし,段数は2段組で段間は2文字とする.
  余白は
  天(\texttt{top})30mm,ノド(\texttt{inner})24mm,小口(\texttt{outor})18mmとし,
  地(\texttt{bottom})が20mm以内に収まるように版面を設定する.

\section{1行の文字数と行数の算出}
  \subsection{長さの単位}
  \LaTeX における$1 pt = \frac{1}{72.27} inch$である.また,$1 inch = 25.4 mm$である.
  また,1Q(H) = 0.25mmである.
  
  \subsection{1行の文字数の算出}
  文字の大きさ12Q(H)であるから,
  \[ 12 \times 0.25 mm = 3 mm \]
  となる.  
  A4の横は$210mm$であるから,左右の余白を引いた版面の横幅は
  $210mm - 24mm - 18mm = 168mm$になればよい.文字の横幅は3mmであるから,56文字分に相当する.
  2段組で段間は2文字であるから各段横27字にすると$ 27 + 2 + 27 = 56 $となり,版面の横幅と等しくなる.

  \subsection{行数の算出}
  \texttt{jsarticle}クラスで12Qオプションを設定した場合の拡大率は$\frac{\the\mag}{1000}$である.

  行送りは\the\baselineskip であるから,
  \[ 16 \times \frac{1}{72.27} inch \times 25.4 mm/inch \times \frac{923}{1000} \approx 5.190 mm \]
  となる.
  % これは約20.760Hである.
  また,行間は
  \[ 5.190 mm - 3 mm = 2.190 mm \]
  と求められる.
  % これは約8.761Hである.
  
  次に,上下の余白を引いた版面の縦幅は$297mm - 30mm - 20mm = 247mm$以内になればよい.
  最後の行の行間$2.190mm$は不要であるので許容される縦幅に予め加え,
  行送り$5.190mm$で割ると
  \[ (247mm + 2.190mm )\div 5.190mm = 48.01 \]
  となり,48行とすれば縦幅に収まることになる.
  
  改めて,
  \[ 5.190mm \times 48 = 249.12mm \]
  であるから,
  最後の行の行間を余白だとすれば
  $297mm -30mm - 249.12 + 2.190mm = 20.07mm$,
  余白だとしなければ
  $297mm -30mm - 249.12 = 17.88mm$が地の部分の余白となる.
  
  実際には最後の行の行間にfootnoteなどの小さい文字が収まってしまい,仕様の20mmをはみ出ることがあることがわかった.
  
  そこで,行数を47行として計算すると,
  
  \[ 5.190mm \times 47 = 243.93mm \]
  であるから,
  最後の行の行間を余白だとすれば
  $297 mm - 30 m - 243.93mm + 2.190 = 25.25 mm$
  が地の部分の余白となる.
  最後の行の行間を余白だとしなければ
  $297 mm - 30 m - 243.93mm = 23.06 mm$
  が地の部分の余白となる.

  よって,この設定では通常は25mmで,footnoteや数式などを挿入して行が押し込まれたときは約23mmまで,地の余白が伸縮する.
  
\section{実際の設定}

  以上の考察から,47行の場合の実際の設定は次のとおりとした.

{\small
\begin{verbatim}
\documentclass[a4j,12Q,twocolumn,twoside]{jsarticle}
\usepackage[
  top=30truemm, inner=24truemm, % 天とノド
  bottom=24.99truemm, outer=18truemm, % 地と小口
  dvipdfm]{geometry}
\end{verbatim}
}

  地には25.26mmないしは25mmを設定したかったが,実際にレンダリングしたところ行数が47行におさまらなくなったのでなるべく近い値として24.99mmを設定した.
  
  なお,行数を48行にする場合,20.07mmを設定したいところだが,19.80mmにすると0.2mmはみでるもののうまく収まる.
  
\section{二分あきおよび四分二分あきの試み}
  baselinestretch に倍率を設定し,行間を二分あき,または,二分四分あきにすることを検討する.
  
  二分あきの場合,
  \[ 12H + 12H/2 \times 0.25mm/H= 4.5mm\]であるから,
  行送りを現在の
  $4.5mm \div 5.190mm \approx 0.867$
  倍すればよい.

  二分四分あきの場合,
  \[ 12H + 12H/2 + 12H/4 \times 0.25mm/H= 5.25mm\]であるから,
  $5.25mm \div 5.190mm \approx 1.0115$倍となり, 
  今の設定よりも若干広くなるものの,ほぼ変わらない.
  実際にこの設定を行ったところ,行間は正しく設定できたものの,大きい文字を使うと
  左右の段が揃わない問題が発生したため,標準の設定を利用することとする.
\section{文字数・行数の確認}
\noindent
1234567890
1234567890
1234567890\footnote{寿限無寿限無五劫の摺り切れ海砂利水魚の水行末雲来末風来末.食う寝る所に住む所藪柑子ブラコウジ.パイポパイポパイポのシューリンガングーリンダイのポンポコピーのポンポコナーの長久命の長助.寿限無寿限無五劫の摺り切れ海砂利水魚の水行末雲来末風来末.}
\newpage
\noindent 1\par \noindent 2\par \noindent 3\par \noindent 4\par \noindent 5\par
\noindent 6\par \noindent 7\par \noindent 8\par \noindent 9\par \noindent 0\par
\noindent 1\par \noindent 2\par \noindent 3\par \noindent 4\par \noindent 5\par
\noindent 6\par \noindent 7\par \noindent 8\par \noindent 9\par \noindent 0\par
\noindent 1\par \noindent 2\par \noindent 3\par \noindent 4\par \noindent 5\par
\noindent 6\par \noindent 7\par \noindent 8\par \noindent 9\par \noindent 0\par
\noindent 1\par \noindent 2\par \noindent 3\par \noindent 4\par \noindent 5\par
\noindent 6\par \noindent 7\par \noindent 8\par \noindent 9\par \noindent 0\par
\noindent 1\par \noindent 2\par \noindent 3\par \noindent 4\par \noindent 5\par
\noindent 6\par \noindent 7\par \noindent 8\par \noindent 9\par \noindent 0\par
%\noindent 1\par %\noindent 2\par \noindent 3\par \noindent 4\par \noindent 5\par

\end{document}
